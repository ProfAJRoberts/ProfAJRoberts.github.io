\documentclass[11pt,a5paper]{article}
\usepackage{amsmath,amsfonts,parskip,enumitem,pgfplots}
\pgfplotsset{compat=newest}
%\usepgfplotslibrary{patchplots}% for "patch refines"
%\usepackage{pgfplotstable} % for some table plots
% view stereo with cross eyes, or vice-versa
%\newcommand{\qview}[2]{\foreach \q in {#2,#1}}
\newcommand{\answer}[1]{} % do nothing with answers
\renewcommand{\vec}[1]{\text{\boldmath$#1$}}
\renewcommand{\Vec}[1]{%
  \expandafter\def\csname#1v\endcsname%
  {\ensuremath{\vec #1}}}

\title{Exercise \jobname}
\author{A. J. Roberts, \today}
\date{}

\begin{document}

\maketitle

\Vec u\Vec v\Vec x
\def\proj{\operatorname{proj}}
\def\Span{\operatorname{span}}
\def\XX{\ensuremath{\mathbb X}}
For the subspace \(\XX=\Span\{\xv\}\) and the vector~\vv, draw the decomposition of~\vv\ into the sum of vectors in~\XX\ and~\(\XX^\perp\).

\newcommand{\projxv}[9]{\begin{tikzpicture}
  \begin{axis}[small
  ,axis equal ,axis x line=none , axis y line=none
  ,samples=2, enlarge x limits={value=0.15,auto} ]
  \addplot[black,mark=*]coordinates {(0,0)};
  \addplot[red,quiver={u=#1,v=#2},-stealth]coordinates {(0,0)};
  \node[right] at (axis cs:#1,#2) {$\vec #9$};
  \addplot[blue,quiver={u=#3,v=#4},-stealth]coordinates {(0,0)};
  \node[right] at (axis cs:#3,#4) {$\vec #8$};
  \ifnum#7>0
  \addplot[black] coordinates {(#5/2,#6/2)} node {proj};
  \addplot[brown,thick,quiver={u=#5,v=#6},-stealth]coordinates {(0,0)};
  \addplot[black] coordinates {(#1/2+#5/2,#2/2+#6/2)} node {perp};
  \addplot[brown,thick,quiver={u=#1-#5,v=#2-#6},-stealth]coordinates {(#5,#6)};
  \fi
  \end{axis}
  \end{tikzpicture}}


\projxv{-1.6}{ 1.2}{1.6}{ -3}{-0.8526}{ 1.5986}0xv


\end{document}
