\documentclass[12pt,a5paper]{article}
\usepackage[a5paper,margin=14mm]{geometry}

\title{The importance of beings fractal}
\author{Prof A.J. Roberts\\
School of Mathematical Sciences\\
University of Adelaide\\
\texttt{mailto:anthony.roberts@adelaide.edu.au}
}
\pagestyle{headings}

\begin{document}

\maketitle


"Good science is the ability to look at things in a new way and 
achieve an understanding that you didn't have before ... It is opening 
windows on the world ... you perceive a little tiny glimpse of the way 
the Universe hangs together, which is a wonderful feeling."  Hans 
Kornberg

Fractal geometry, largely inspired by Benoit Mandelbrot during the 
sixties and seventies, is one of the great advances in mathematics 
for two thousand years.  Given the rich and diverse power of 
developments in mathematics and its applications, this is a 
remarkable claim.  Often presented as being just a part of modern 
chaos theory, fractals are momentous in their own right.  Euclid's 
geometry describes the world around us in terms of points, lines and 
planes---for two thousand years these have formed the somewhat 
limited repertoire of basic geometric objects with which to describe 
the universe.  \emph{Fractals immeasurably enhance this world-view by 
providing a description of much around us that is rough and 
fragmented---of objects that have structure on many sizes.}  Examples 
include: coastlines, rivers, plant distributions, architecture, wind 
gusts, music, and the cardiovascular system.  

\section{Some fractal models}

Before discussing in detail the common feature of the previously 
mentioned examples, I present a few examples of fractals and the type 
of physical applications that they have.  

\subsection{Noise and natural events}

Have you ever noticed that there are some days where nothing goes 
right?  times when you just cannot get a decent telephone 
connection?  years when drought follows drought?  long periods when 
gusts of wind come thick and fast?  That events often occur in bursts 
is a well documented aspect of the world.  The Cantor set is a model 
for such bursty phenomena.  Construct the Cantor set in the following 
manner:
start with a bar of some length; then remove its middle third to 
leave two separate thirds; then remove the middle thirds of these to 
leave four separate ninths; then remove the middle thirds of these to 
obtain eight separate twenty-sevenths; and so on.  Eventually we just 
obtain a scattered dust of points.  However, this dust is specially 
distributed into pairs of points, of pairs of pairs of points, and so 
on.  If the original bar represented a time interval, then the dust 
represents times when events occur and the striking feature is that 
there are long quiescent periods separating the short bursts of 
activity that a clump of the points represents.

\subsection{Coastlines and rivers}

The line of a coast or the path of a river is tortuous.  On a 
small-scale map of Australia or any other country the coastline has 
lots of wriggles.  Upon examining a larger-scale map the wriggles 
will be resolved clearly into gulfs and peninsulas.  However, many 
smaller scale wriggles will still be seen.  These can only be 
resolved by looking at an even larger scale map, whereupon they will 
be seen to be inlets and spits.  But once again there will be 
wriggles in the coast.  Similarly for rivers---they exhibit bends and 
meanders upon many scales of length.  The Koch curve models these 
phenomena.

Starting with an equilateral triangle, we replace the middle third of 
each side by two segments of the same length (as if we pasted an 
equilateral triangle of one-third the size onto each side); this 
forms the second picture above showing large-scale peninsulas and 
bays.  Repeating this process of extracting the middle third of each 
straight side and replacing it with two segments of the same length, 
the next stage of the construction gives the third picture; it shows 
smaller inlets and spits.  Continually repeating this process leads 
to a very wriggly line that is the Koch curve.  It is perhaps too 
convoluted for a coastline, but on the other hand, it looks far more 
realistic than a routine curve!

\subsection{Turbulence}

Most mathematicians, physicists and engineers would give their right 
arm to understand what is going on in this picture.  It shows 
something of the highly complex motion that is turbulence in a fluid 
as expressed by the following ditty by L.F. Richardson:
Big whorls have little whorls,
Which feed on their velocity;
And little whorls have lesser whorls,
and so on to viscosity.

When air or water moves, a smooth flow quickly breaks up into 
swirling eddies.  These eddies are of a wide range of sizes and, as 
on a windy day, there are often quiescent periods separating the 
various wind gusts.  The structure of turbulence may be epitomised by 
a Sierpinski sponge which is formed from building blocks in much the 
same way as the "Eiffel tower."  Form a small unit by putting 20 
blocks face to face along the edges and corners of a 3x3x3 cube, 
leaving the middle of the six faces and the middle of the cube 
vacant.  Make a bigger unit by connecting 20 of these units together 
in the same 3x3x3 pattern.  And so on to as large a scale as possible.

\section{Scaling and dimensionality}

The common theme in these examples is not just that they have detail 
on many lengths, but also that the structure at any scale is much the 
same at any other scale---the coastline around a continent looks just 
like any small part of the coastline.  If we take a magnifying glass 
or microscope to any of these examples then, no matter what the 
magnification, the geometric detail that we see is the same.  This 
property of looking similar at all scales is termed self-similarity: 
exact in the artificial examples, and statistical or random in 
practical applications.  In order to tease out the self-similar 
characteristics of such objects we need to explore the fractal over 
many lengths and sizes.  

The coarsest characteristic of fractals is their dimensionality.  
While we normally expect a dimension to be an integer, a natural 
number such as 1, 2 or 3, fractals are best described by means of a 
dimension which is fractional, such as 1.2 or 0.69.  This dimension 
is obtained by blurring the fractal at some size, counting the number 
of blobs in this blurred picture, and then seeing how the count 
varies with the size of the blurring.  Another explanation of this 
process is to count how few "clumps" of a certain size the object can 
be broken into, and then see how this count varies with the size of 
the clump.  

\subsection{Points, lines and planes}

Lets become familiar with the argument via some well known geometric 
examples: points, lines, and planes.  Consider a line of some length 
L as shown below.  The line
could be curved, but for simplicity we take it to be straight.  To 
"blur" the object at a size d I mean that we try to cover as much of 
the object as possible by discs of diameter d.  In this picture I 
have used N=9 discs to cover the line segment.  If the discs were 
half the diameter, then we would have to use twice as many of them to 
cover the line;  if the discs were one-third the diameter then we 
would have to use three times as many to cover the line;  and so on 
for other sizes.  Typically, the number of discs needed to cover a 
line of length L is N=L/d.  The important aspect of this relation is 
that the number of discs is inversely proportional to the first power 
of the size (diameter) of the discs: .  

Contrast this with what happens when we cover an area A of the plane 
with discs of some diameter d: N=34 in the above example.  Typically, 
the number of discs needed to cover an area A is inversely 
proportional to the second power of the size of the discs: ;  the 
number would be close to the area, A, divided by the area of each 
disc, , to be  if it were not for the wastage around the perimeter 
of each disc.  

See that the exponent of this relation between N, the number of 
discs, and the size of the discs, as measured by the diameter d, is 
precisely the dimensionality of the object:  a line is 
one-dimensional;  an area of the plane is two-dimensional.  This 
relation between the exponent and the dimensionality is true in 
general.  For another example, consider a small number, n, of points 
distributed in space---for all d smaller than the minimum separation 
between the points the number of discs needed is precisely the same 
as the number of points in the set.  Thus  , and the 0�exponent 
matches the zero-dimensionality of a point or a finite number of 
points.  

For the geometric objects introduced earlier, the relation between N 
and d involves a fractional exponent�D.  It is only reasonable for us 
to say that the dimensionality of such an object is the fraction�D.

\end{document}

