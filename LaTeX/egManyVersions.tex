\documentclass[11pt,a4paper]{article}

\title{Example of a question with various versions generated randomly via the package~fp}
\author{Tony Roberts}
\date{\Hour:\Minutes, 8 September 2018}

\usepackage{amsmath}
\usepackage[nomessages]{fp}
% prints variable or expression to some digits, four by default, clipping zeros.
\newcommand{\FP}[2][4]{\FPeval\TempRes{clip(round(#2:#1))}\FPprint\TempRes}
% prints variable with parentheses if negative, to some digits, clipping
\newcommand{\FPP}[2][4]{\FPifneg#2(\FP[#1]#2)\else\FP[#1]#2\fi}

% in order to iterate with foreach, use this or pgfplots
\usepackage{pgffor}

\FPeval\Hour{trunc(\the\time/60+0.001:0)}
\FPeval\Minutes{round(\the\time-60*\Hour:0)}
\newenvironment{solution}{\paragraph{Solution}\small}{}
\def\xv{\ensuremath{\vec x}} \def\yv{\ensuremath{\vec y}}
\def\uv{\ensuremath{\vec u}} \def\vv{\ensuremath{\vec v}}


\begin{document}
\maketitle


\section{Example question}

The following question changes every minute because I set a random seed from the number of minutes so far in this day.  In practice, instead set a random seed which includes ``\verb|\the\year|'' so the question changes from year to year.
\FPseed=123\the\time % set the random seed
\FPrandom\tmp % generate a first random value to get away from small

\begin{enumerate}
\item Consider the orthogonal matrix 
\begin{equation*}
Q=\frac12\begin{bmatrix}    
   1&1&1&1\\
   1&1&-1&-1\\
   1&-1&1&-1\\
   1&-1&-1&1
 \end{bmatrix}.
\end{equation*}

\begin{enumerate}
% generate random integer four-vector x and u=Qx
\FPrandom\tmp \FPeval\xa{round(5*\tmp:0)}
\FPrandom\tmp \FPeval\xb{round(5*\tmp:0)}
\FPrandom\tmp \FPeval\xc{round(5*\tmp:0)}
\FPrandom\tmp \FPeval\xd{round(5*\tmp:0)}
\FPeval\ua{(\xa+\xb+\xc+\xd)/2}
\FPeval\ub{(\xa+\xb-\xc-\xd)/2}
\FPeval\uc{(\xa-\xb+\xc-\xd)/2}
\FPeval\ud{(\xa-\xb-\xc+\xd)/2}
\item Let $\xv =(\xa,\xb,\xc,\xd)$, 
compute $\uv =Q\xv $ and verify that $\| \xv \|=\| \uv \|$.  
\begin{solution} 
$\| \xv \|^2 = \xa^2 + \xb^2 + \xc^2 + \xd^2 
= \FP{\xa^2+\xb^2+\xc^2+\xd^2}$.
\begin{align*}
& \uv =Q\xv
= \frac12\begin{bmatrix}    
   1&1&1&1\\
   1&1&-1&-1\\
   1&-1&1&-1\\
   1&-1&-1&1
 \end{bmatrix}
 \begin{bmatrix} \xa\\\xb\\\xc\\\xd \end{bmatrix}
= \begin{bmatrix} \FP\ua\\ \FP\ub\\ \FP\uc\\ \FP\ud \end{bmatrix}
\\&
\implies \| \uv \|^2 
= \FPP\ua^2 +\FPP\ub^2 +\FPP\uc^2 +\FPP\ud^2 
= \FP{\ua*\ua+\ub*\ub+\uc*\uc+\ud*\ud} 
= \| \xv \|^2.
\end{align*}
\end{solution}

% generate random integer four-vector y and v=Qy
\FPrandom\tmp \FPeval\ya{round(9*\tmp-4:0)}
\FPrandom\tmp \FPeval\yb{round(9*\tmp-4:0)}
\FPrandom\tmp \FPeval\yc{round(9*\tmp-4:0)}
\FPrandom\tmp \FPeval\yd{round(9*\tmp-4:0)}
\FPeval\va{(\ya+\yb+\yc+\yd)/2}
\FPeval\vb{(\ya+\yb-\yc-\yd)/2}
\FPeval\vc{(\ya-\yb+\yc-\yd)/2}
\FPeval\vd{(\ya-\yb-\yc+\yd)/2}
\item Let $\yv =(\ya,\yb,\yc,\yd)$, compute $\vv =Q\yv $ and verify that $\| \yv \|=\| \vv \|$.
\begin{solution} 
$\| \yv \|^2 
= \FPP\ya^2 +\FPP\yb^2 +\FPP\yc^2 +\FPP\yd^2
% Occasionally get cryptic failure that is fixed by such extra parentheses!
= \FP{(\ya*\ya+\yb*\yb+\yc*\yc+\yd*\yd)}$.
\begin{align*}
& \vv =Q\yv
= \frac12\begin{bmatrix}    
   1&1&1&1\\
   1&1&-1&-1\\
   1&-1&1&-1\\
   1&-1&-1&1
 \end{bmatrix}
 \begin{bmatrix} \ya\\\yb\\\yc\\\yd \end{bmatrix}
= \begin{bmatrix} \FP\va\\\FP\vb\\\FP\vc\\\FP\vd \end{bmatrix}
\\&
\implies \|\vv \|^2
= \FPP\va^2 +\FPP\vb^2 +\FPP\vc^2 +\FPP\vd^2 
= \FP{\va*\va+\vb*\vb+\vc*\vc+\vd*\vd}
= \|\yv \|^2. 
\end{align*}
\end{solution}

\item Use the dot product to confirm that the angle between~\xv\ and~\yv\ is the same as that between~\uv\ and~\vv.
\begin{solution} 
$\xv \cdot \yv 
= \xv ^T \yv 
= \begin{bmatrix} \xa&\xb&\xc&\xd \end{bmatrix}
  \begin{bmatrix} \ya\\\yb\\\yc\\\yd \end{bmatrix}
= \FP{\xa*\ya+\xb*\yb+\xc*\yc+\xd*\yd}.$ 

$ \uv \cdot \vv 
= \uv ^T \vv 
= \begin{bmatrix} \FP\ua&\FP\ub&\FP\uc&\FP\ud \end{bmatrix}
  \begin{bmatrix} \FP\va\\\FP\vb\\\FP\vc\\\FP\vd \end{bmatrix}
= \FP{\ua*\va+\ub*\vb+\uc*\vc+\ud*\vd} 
= \xv \cdot \yv $.

Since these dot products are the same, and \(\|\uv \|=\|\xv \|\) and \(\|\vv \|=\|\yv \|\), the angle between~\xv\ and~\yv\ is the same as that between~\uv\ and~\vv. 
\end{solution}
\end{enumerate}
\end{enumerate}




\section{Useful fp information}
\begin{itemize}
\item \verb|\usepackage[nomessages]{fp}| in the preamble.
This package was last revised in 1999.

\item Also define a useful command to print any number that internally may not be an integer:
\begin{verbatim}
\newcommand{\FP}[2][4]{%
    \FPeval\TempRes{clip(round(#2:#1))}%
    \FPprint\TempRes}
\end{verbatim}
\verb|\FP\x| or \verb|\FP{expression}| prints the variable\slash expression rounded to four decimal places, and clipped to remove trailing zeros;  \verb|\FP[n]...| rounds to \verb|n|~decimal places.
Even if the result of a computation should be an integer, the fixed point arithmetic used by~\verb|fp| often causes error.
Do most printing via this command, and thereby keep full accuracy for internal variables.

\item And also useful to define this command that uses~\verb|\FP| to print a variable (not an expression) with parentheses when negative, and without parentheses when positive or zero.
\begin{verbatim}
\newcommand{\FPP}[2][4]{%
    \FPifneg#2(\FP[#1]#2)\else\FP[#1]#2\fi}
\end{verbatim}


\item The package~\verb|fp| does fixed point arithmetic on signed decimal numbers up to~\(10^{18}\), with a fixed resolution of 18~decimal places.



\item \verb|\FPeval#1{#2}| assigns to \verb|#1| the value computed by the expression~\verb|#2|.  
Occasionally I get a cryptic error message that is fixed by inserting an extra pair of parentheses: \verb|\FPeval#1{(#2)}|.

\item Defined infix operations for evaluation are: \verb|+|, \verb|-|, \verb|*|, \verb|/|, \verb|^| for add, sub, mul, div, pow.
The unary minus is not defined: use \verb|neg()|.

\item Defined functions include abs, neg, min, max, 
	   round, trunc, clip, exp, ln, pow, root, sin, cos, 
	   tan, cot, arcsin, arccos, arctan, arccot 
	   
Note: trig functions use radians; \quad
\(\verb|root(a,b)|=\sqrt[a]b\) so a square-root is \verb|root(2,b)|;  \quad
\(\verb|pow(a,b)|=b^a=\verb|exp(a*ln(b))|\) so fails for negative~\verb|b| (even if \(\verb|a|=2\)); \quad
\verb|round(a:n)| has the value of~\verb|a| rounded to \verb|n|~decimal places; \quad
\verb|clip(a)| has the value of~\verb|a| but with trailing zeros clipped.

\item Defined constants are~\verb|pi| and~\verb|e|.

\item \verb|\FPrandom#1| assigns to \verb|#1| a random number between~0 and~1.

When scaling and rounding to include random negative integers, sometimes get~\verb|-0| which appears poorly: however, \verb|\FP...| prints the value~\verb|-0| properly as~\verb|0|.

\item \verb|\FPseed=#1| where \verb|#1| is some string of digits sets the seed for the random number generator.
For example, \verb|\FPseed=67\the\year| is this year the same as \verb|\FPseed=672018| and so sets the seed unique to the problem via~67, and unique to the year via \verb|\the\year|.

\item These are some of the conditional statements:
\begin{itemize}
\item \verb|\FPifneg#1 ...\else...\fi| \quad tests \(\verb|#1| < 0\)\,?
\item \verb|\FPiflt#1#2...\else...\fi| \quad tests \(\verb|#1| < \verb|#2|\)\,?
\item \verb|\FPifeq#1#2...\else...\fi| \quad tests \(\verb|#1| = \verb|#2|\)\,?
\item \verb|\FPifgt#1#2...\else...\fi| \quad tests \(\verb|#1| > \verb|#2|\)\,?
\end{itemize}

\end{itemize}





\section{More examples}

\paragraph{To iterate} 
One may use \verb|\foreach| from \verb|\usepackage{pgffor}| (or \verb|pgfplots|).  For example, the following iterates the map \(x_{k}=\cos x_{k-1}\) ten times via the mysterious magic of \verb|\xdef|.
\FPeval\xk{0}%
\foreach \k in {1,2,...,10}%
    {\FPeval\xNext{cos(\xk)}\xdef\xk{\xNext}}%
The following code gives ``From \(x_0=0\), iteration gives \(x_{10}=\FP\xk\).''
\begin{verbatim}
\FPeval\xk{0}
\foreach \k in {1,2,...,10} 
    {\FPeval\xNext{cos(\xk)}\xdef\xk{\xNext}}
From \(x_0=0\), iteration gives \(x_{10}=\FP\xk\).
\end{verbatim}

One could just typeset some of the iterates from within the loop:
``\FPeval\xk{0}the iterations are \(x_0=\FP\xk\)%
\foreach \k in {1,2,...,20}%
    {\FPeval\xNext{cos(\xk)}\xdef\xk{\xNext}%
     \ifnum\k<4, \(x_{\k}=\FP\xk\)\fi}%end-for
    , and so on to  \(x_{20}=\FP\xk\).''
Typeset this with the code
\begin{verbatim}
\FPeval\xk{0}the iterations are \(x_0=\FP\xk\)%
\foreach \k in {1,2,...,20}%
    {\FPeval\xNext{cos(\xk)}\xdef\xk{\xNext}%
     \ifnum\k<4, \(x_{\k}=\FP\xk\)\fi}%end-for
    , and so on to  \(x_{20}=\FP\xk\).
\end{verbatim}





\end{document}
