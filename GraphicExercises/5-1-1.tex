\documentclass[11pt,a5paper]{article}
\usepackage{amsmath,parskip,enumitem,pgfplots}
\pgfplotsset{compat=newest}
%\usepgfplotslibrary{patchplots}% for "patch refines"
%\usepackage{pgfplotstable} % for some table plots
% view stereo with cross eyes, or vice-versa
%\newcommand{\qview}[2]{\foreach \q in {#2,#1}}
\newcommand{\answer}[1]{} % do nothing with answers
\renewcommand{\vec}[1]{\text{\boldmath$#1$}}
\renewcommand{\Vec}[1]{%
  \expandafter\def\csname#1v\endcsname%
  {\ensuremath{\vec #1}}}


\title{Exercise \jobname}
\author{A. J. Roberts, \today}
\date{}

\begin{document}

\maketitle

\Vec x\Vec v
For some distinct \(2\times2\) matrices~\(A\) the following plots adjoin the product \(A\xv\) to~\xv\ for a complete range of unit vectors~\xv.
Use each plot to roughly estimate the norm of the underlying matrix~\(A\) for that plot.

\newcommand{\eRose}[4]{\begin{tikzpicture}%
    %[baseline={([yshift={-\ht\strutbox}]current bounding box.north)}] 
    \begin{axis}[small,font=\footnotesize
        ,axis equal image, axis lines=middle
        ,samples=33] %25 or 33 not bad 
        \addplot[domain=0:360,quiver={u=cos(\x),v=sin(\x)},blue,-stealth] 
        ({0},{0});
        \addplot[domain=0:360,quiver={u=#1*x+#2*y,v=#3*x+#4*y},red,-stealth] 
        ({cos(\x)},{sin(\x)});
    \end{axis}
    \end{tikzpicture}}


\eRose{-1.0}{0.8}{1.2}{0.5}
\answer{1.6}





\end{document}
