\documentclass[11pt,a5paper]{article}
\usepackage{amsmath,parskip,pgfplots}
\pgfplotsset{compat=newest}
%\usepgfplotslibrary{patchplots}% for "patch refines"
%\usepackage{pgfplotstable} % for some table plots
% view stereo with cross eyes, or vice-versa
%\newcommand{\qview}[2]{\foreach \q in {#2,#1}}
\newcommand{\answer}[1]{} % do nothing with answers
\renewcommand{\vec}[1]{\text{\boldmath$#1$}}
\renewcommand{\Vec}[1]{%
  \expandafter\def\csname#1v\endcsname%
  {\ensuremath{\vec #1}}}


\title{Exercise \jobname}
\author{A. J. Roberts, \today}
\date{}

\begin{document}

\maketitle

For each of the given illustrations of a linear transformation of the unit square, `guesstimate' by eye the determinant of the matrix of the transformation (estimate to an error of say~33\% \text{or so).}


\newcommand{\TwoD}[4]{%
    \pgfmathparse{abs(#1)+abs(#2)+abs(#3)+abs(#4)}%
    \pgfmathparse{(\pgfmathresult>5)+(\pgfmathresult>3.6)*0.5+0.5}%
    \edef\xtdst{\pgfmathresult}%
    \begin{tikzpicture}
    \begin{axis}[small,font=\footnotesize
    ,xtick distance={\xtdst}
      ,axis lines=middle,thick,axis equal image, enlargelimits=0.05 %,grid,no marks
      ] 
    \addplot coordinates {(0,1)(1,1)(1,0)(0,0)(0,1)(0.5,1.2)(1,1)}; 
    \addplot coordinates {(#2,#4)(#1+#2,#3+#4)(#1,#3)(0,0)
      (#2,#4)(#1*0.5+#2*1.2,#3*0.5+#4*1.2)(#1+#2,#3+#4)}; 
    \end{axis}
    \end{tikzpicture}
}


\TwoD{0.4}{0.8}{-1.2}{0.7}
\answer{\(\det\approx 1.24\)}

\end{document}
